% Options for packages loaded elsewhere
\PassOptionsToPackage{unicode}{hyperref}
\PassOptionsToPackage{hyphens}{url}
%
\documentclass[
]{article}
\usepackage{amsmath,amssymb}
\usepackage{iftex}
\ifPDFTeX
  \usepackage[T1]{fontenc}
  \usepackage[utf8]{inputenc}
  \usepackage{textcomp} % provide euro and other symbols
\else % if luatex or xetex
  \usepackage{unicode-math} % this also loads fontspec
  \defaultfontfeatures{Scale=MatchLowercase}
  \defaultfontfeatures[\rmfamily]{Ligatures=TeX,Scale=1}
\fi
\usepackage{lmodern}
\ifPDFTeX\else
  % xetex/luatex font selection
\fi
% Use upquote if available, for straight quotes in verbatim environments
\IfFileExists{upquote.sty}{\usepackage{upquote}}{}
\IfFileExists{microtype.sty}{% use microtype if available
  \usepackage[]{microtype}
  \UseMicrotypeSet[protrusion]{basicmath} % disable protrusion for tt fonts
}{}
\makeatletter
\@ifundefined{KOMAClassName}{% if non-KOMA class
  \IfFileExists{parskip.sty}{%
    \usepackage{parskip}
  }{% else
    \setlength{\parindent}{0pt}
    \setlength{\parskip}{6pt plus 2pt minus 1pt}}
}{% if KOMA class
  \KOMAoptions{parskip=half}}
\makeatother
\usepackage{xcolor}
\usepackage[margin=1in]{geometry}
\usepackage{color}
\usepackage{fancyvrb}
\newcommand{\VerbBar}{|}
\newcommand{\VERB}{\Verb[commandchars=\\\{\}]}
\DefineVerbatimEnvironment{Highlighting}{Verbatim}{commandchars=\\\{\}}
% Add ',fontsize=\small' for more characters per line
\usepackage{framed}
\definecolor{shadecolor}{RGB}{248,248,248}
\newenvironment{Shaded}{\begin{snugshade}}{\end{snugshade}}
\newcommand{\AlertTok}[1]{\textcolor[rgb]{0.94,0.16,0.16}{#1}}
\newcommand{\AnnotationTok}[1]{\textcolor[rgb]{0.56,0.35,0.01}{\textbf{\textit{#1}}}}
\newcommand{\AttributeTok}[1]{\textcolor[rgb]{0.13,0.29,0.53}{#1}}
\newcommand{\BaseNTok}[1]{\textcolor[rgb]{0.00,0.00,0.81}{#1}}
\newcommand{\BuiltInTok}[1]{#1}
\newcommand{\CharTok}[1]{\textcolor[rgb]{0.31,0.60,0.02}{#1}}
\newcommand{\CommentTok}[1]{\textcolor[rgb]{0.56,0.35,0.01}{\textit{#1}}}
\newcommand{\CommentVarTok}[1]{\textcolor[rgb]{0.56,0.35,0.01}{\textbf{\textit{#1}}}}
\newcommand{\ConstantTok}[1]{\textcolor[rgb]{0.56,0.35,0.01}{#1}}
\newcommand{\ControlFlowTok}[1]{\textcolor[rgb]{0.13,0.29,0.53}{\textbf{#1}}}
\newcommand{\DataTypeTok}[1]{\textcolor[rgb]{0.13,0.29,0.53}{#1}}
\newcommand{\DecValTok}[1]{\textcolor[rgb]{0.00,0.00,0.81}{#1}}
\newcommand{\DocumentationTok}[1]{\textcolor[rgb]{0.56,0.35,0.01}{\textbf{\textit{#1}}}}
\newcommand{\ErrorTok}[1]{\textcolor[rgb]{0.64,0.00,0.00}{\textbf{#1}}}
\newcommand{\ExtensionTok}[1]{#1}
\newcommand{\FloatTok}[1]{\textcolor[rgb]{0.00,0.00,0.81}{#1}}
\newcommand{\FunctionTok}[1]{\textcolor[rgb]{0.13,0.29,0.53}{\textbf{#1}}}
\newcommand{\ImportTok}[1]{#1}
\newcommand{\InformationTok}[1]{\textcolor[rgb]{0.56,0.35,0.01}{\textbf{\textit{#1}}}}
\newcommand{\KeywordTok}[1]{\textcolor[rgb]{0.13,0.29,0.53}{\textbf{#1}}}
\newcommand{\NormalTok}[1]{#1}
\newcommand{\OperatorTok}[1]{\textcolor[rgb]{0.81,0.36,0.00}{\textbf{#1}}}
\newcommand{\OtherTok}[1]{\textcolor[rgb]{0.56,0.35,0.01}{#1}}
\newcommand{\PreprocessorTok}[1]{\textcolor[rgb]{0.56,0.35,0.01}{\textit{#1}}}
\newcommand{\RegionMarkerTok}[1]{#1}
\newcommand{\SpecialCharTok}[1]{\textcolor[rgb]{0.81,0.36,0.00}{\textbf{#1}}}
\newcommand{\SpecialStringTok}[1]{\textcolor[rgb]{0.31,0.60,0.02}{#1}}
\newcommand{\StringTok}[1]{\textcolor[rgb]{0.31,0.60,0.02}{#1}}
\newcommand{\VariableTok}[1]{\textcolor[rgb]{0.00,0.00,0.00}{#1}}
\newcommand{\VerbatimStringTok}[1]{\textcolor[rgb]{0.31,0.60,0.02}{#1}}
\newcommand{\WarningTok}[1]{\textcolor[rgb]{0.56,0.35,0.01}{\textbf{\textit{#1}}}}
\usepackage{graphicx}
\makeatletter
\def\maxwidth{\ifdim\Gin@nat@width>\linewidth\linewidth\else\Gin@nat@width\fi}
\def\maxheight{\ifdim\Gin@nat@height>\textheight\textheight\else\Gin@nat@height\fi}
\makeatother
% Scale images if necessary, so that they will not overflow the page
% margins by default, and it is still possible to overwrite the defaults
% using explicit options in \includegraphics[width, height, ...]{}
\setkeys{Gin}{width=\maxwidth,height=\maxheight,keepaspectratio}
% Set default figure placement to htbp
\makeatletter
\def\fps@figure{htbp}
\makeatother
\setlength{\emergencystretch}{3em} % prevent overfull lines
\providecommand{\tightlist}{%
  \setlength{\itemsep}{0pt}\setlength{\parskip}{0pt}}
\setcounter{secnumdepth}{-\maxdimen} % remove section numbering
\ifLuaTeX
  \usepackage{selnolig}  % disable illegal ligatures
\fi
\usepackage{bookmark}
\IfFileExists{xurl.sty}{\usepackage{xurl}}{} % add URL line breaks if available
\urlstyle{same}
\hypersetup{
  pdftitle={lme4 cheat sheet},
  pdfauthor={Clay Ford},
  hidelinks,
  pdfcreator={LaTeX via pandoc}}

\title{lme4 cheat sheet}
\author{Clay Ford}
\date{Fall 2015}

\begin{document}
\maketitle

A cheat sheet for fitting and assessing Linear Mixed-Effect Models using
\texttt{lme4}.

\section{\texorpdfstring{Limitations of \texttt{lme4} (or what it cannot
do that \texttt{nlme} can
do)}{Limitations of lme4 (or what it cannot do that nlme can do)}}\label{limitations-of-lme4-or-what-it-cannot-do-that-nlme-can-do}

\begin{itemize}
\item
  \texttt{lme4} does not allow the modeling of heteroscedastic
  within-group errors. It only fits models with independent residual
  errors.
\item
  \texttt{lme4} only models two types of covariance matrices for random
  effects: general and diagonal. A general covariance matrix has
  separate variances for each random effect and covariances between the
  random effects. A diagonal covariance matrix has separate variances
  for each random effect and no covariance between random effects.
\end{itemize}

\section{\texorpdfstring{Features of \texttt{lme4} (or what it can do
that \texttt{nlme} cannot
do)}{Features of lme4 (or what it can do that nlme cannot do)}}\label{features-of-lme4-or-what-it-can-do-that-nlme-cannot-do}

\begin{itemize}
\tightlist
\item
  \texttt{lme4} provides facilites for modeling generalized linear
  mixed-effects models. (see \texttt{glmer})
\item
  \texttt{lme4} can fit models with \emph{crossed} random effects.
\item
  \texttt{lme4} uses efficient computational algorithms based on
  sparse-matrix representations that make it suitable for large data
  sets.
\end{itemize}

\section{Format of Data}\label{format-of-data}

Data should be a data frame in long format. That is, one row per
observation of the response variable. For example:

\begin{Shaded}
\begin{Highlighting}[]
\FunctionTok{head}\NormalTok{(lme4}\SpecialCharTok{::}\NormalTok{cake)}
\end{Highlighting}
\end{Shaded}

\begin{verbatim}
##   replicate recipe temperature angle temp
## 1         1      A         175    42  175
## 2         1      A         185    46  185
## 3         1      A         195    47  195
## 4         1      A         205    39  205
## 5         1      A         215    53  215
## 6         1      A         225    42  225
\end{verbatim}

\section{Fitting Models}\label{fitting-models}

To fit linear mixed-effects model, use the \texttt{lmer()} function. The
formular for \texttt{lmer} allows you to express both fixed and random
effects. Random effects are defined in parentheses. Random effects are
conditioned on groups, typically groups with uninteresting or
\texttt{random} levels. The conditioning is defined with a pipe:
\texttt{\textbar{}}. Two pipes, \texttt{\textbar{}\textbar{}}, specify
fitting a model with a diagonal covariance structure for the random
effects. (i.e., assume multiple random effects are not correlated.)

Below, \texttt{dv} is the dependent variable, \texttt{x1} is a
predictor, and \texttt{g} is a grouping indicator (for example, subject
ID)

\textbf{Random intercept for each level of \texttt{g}}:
\texttt{lme01\ \textless{}-\ lmer(dv\ \textasciitilde{}\ x1\ +\ (1\ \textbar{}\ g),\ data=df)}

\textbf{Random slope for each level of \texttt{g}}:
\texttt{lme02\ \textless{}-\ lmer(dv\ \textasciitilde{}\ x1\ +\ (0\ +\ x1\ \textbar{}\ g),\ data=df)}

\textbf{Correlated random slope and intercept for each level of
\texttt{g}}:
\texttt{lme03\ \textless{}-\ lmer(dv\ \textasciitilde{}\ x1\ +\ (x1\ \textbar{}\ g),\ data=df)}

\textbf{Uncorrelated random slope and intercept for each level of
\texttt{g}}:
\texttt{lme04\ \textless{}-\ lmer(dv\ \textasciitilde{}\ x1\ +\ (x1\ \textbar{}\textbar{}\ g),\ data=df)}

\subsection{Multilevel models}\label{multilevel-models}

For models with nested grouping factors (aka multilevel models), use
\texttt{/} to indicate nesting. Below \texttt{tch} is nested in
\texttt{sch}. For example, teachers nested within schools.

\textbf{Random intercept for each level of \texttt{sch} and for each
level of \texttt{tch} in \texttt{sch} }:
\texttt{lme01\ \textless{}-\ lmer(dv\ \textasciitilde{}\ x1\ +\ (1\ \textbar{}\ sch/tch),\ data=df)}

\textbf{Random slope for each level of \texttt{sch} and for each level
of \texttt{tch} in \texttt{sch}}:
\texttt{lme02\ \textless{}-\ lmer(dv\ \textasciitilde{}\ x1\ +\ (0\ +\ x1\ \textbar{}\ sch/tch),\ data=df)}

\textbf{Correlated random slope and intercept for each level of
\texttt{sch} and for each level of \texttt{tch} in \texttt{sch}}:
\texttt{lme03\ \textless{}-\ lmer(dv\ \textasciitilde{}\ x1\ +\ (x1\ \textbar{}\ sch/tch),\ data=df)}

\textbf{Uncorrelated random slope and intercept for each level of
\texttt{sch} and for each level of \texttt{tch} in \texttt{sch}}:
\texttt{lme03\ \textless{}-\ lmer(dv\ \textasciitilde{}\ x1\ +\ (x1\ \textbar{}\textbar{}\ sch/tch),\ data=df)}

\section{Extracting and viewing model
information}\label{extracting-and-viewing-model-information}

Say your model is saved as object \texttt{lme01}

\begin{itemize}
\tightlist
\item
  \texttt{summary(lme01)} - View summary of \texttt{lme01}
\item
  \texttt{fixef(lme01)} - View estimated fixed effect coefficients
\item
  \texttt{ranef(lme01)} - View predicted random effects
\item
  \texttt{coef(lme01)} - View coefficients for LMM for \emph{each group}

  \begin{itemize}
  \tightlist
  \item
    \texttt{VarCorr(lme01)} - View estimated variance parameters
  \end{itemize}
\item
  \texttt{confint(lme01)} - Compute confidence intervals on the
  parameters (cutoffs based on the likelihood ratio test)
\item
  \texttt{confint(lme1,\ method="boot")} = Compute confidence intervales
  on the parameters (computed from the bootstrap distribution)
\item
  \texttt{anova(lme1)} - Assess significance of fixed-effect factors
\item
  \texttt{predict(lme1)} - View within-group fitted values for
  \texttt{lme01}
\item
  \texttt{predict(lme1,\ re.form=NA)} - View population fitted values
  for \texttt{lme01}
\end{itemize}

\section{Diagnostic plots}\label{diagnostic-plots}

Say we fit the folllowing LMM:
\texttt{lme1\ \textless{}-\ lmer(y\ \textasciitilde{}\ x\ +\ (x\ \textbar{}\ id),\ data=df)}

\subsection{Within-group errors}\label{within-group-errors}

Plots for examining the assumption that within-group errors are normally
distributed, centered at 0, and have constant variance.

\textbf{standarized residuals vs fitted values (is the scatter
uniform?)}: \texttt{plot(lme1)}

\textbf{standardized residuals versus fitted values by x (is the scatter
uniform within groups?)}:
\texttt{plot(lme1,\ form\ =\ resid(.)\ \textasciitilde{}\ fitted(.)\ \textbar{}\ x)}

\textbf{box-plots of residuals by id (are they centered at 0?)}:
\texttt{plot(lme1,\ form\ =\ id\ \textasciitilde{}\ resid(.))}

\textbf{To assess normality of residuals (does plot seem to lie on
straight line?)}: \texttt{qqnorm(resid(lme1))}

\subsection{Random effects}\label{random-effects}

Plots for examining the assumption that random effects are normally
distributed, centered at 0, and have constant variance.

\textbf{To check constant variance of random effects (is the scatter
uniform?)}: \texttt{plot(ranef(lme1))}

\textbf{To assess normality of random effects (does plot seem to lie on
straight line?)}: \texttt{lattice::qqmath(ranef(lmeEng2))}

\subsection{Model fit}\label{model-fit}

\textbf{observed versus fitted values by id (check fit of model)}:
\texttt{plot(lme1,\ y\ \textasciitilde{}\ fitted(.)\ \textbar{}\ id)}

\section{Comparing models}\label{comparing-models}

When it comes to hypothesis testing, the choice of ML vs REML estimation
is important:

\begin{itemize}
\tightlist
\item
  To compare two models fit by \textbf{REML}, each must have the same
  fixed effects.
\item
  To compare two models fit by \textbf{ML}, one must be nested within
  the other.
\end{itemize}

In \texttt{lmer}, REML is the default. Set \texttt{REML=FALSE} to use ML
estimation.

\subsection{Comparing nested models}\label{comparing-nested-models}

Say we fit two models:
\texttt{lme1\ \textless{}-\ lmer(y\ \textasciitilde{}\ x\ +\ z\ +\ x:z\ +\ (1\ \textbar{}\ g),\ data=df)}
\texttt{lme2\ \textless{}-\ lmer(y\ \textasciitilde{}\ x\ +\ z\ +\ (1\ \textbar{}\ g),\ data=df)}

The following refits the models with ML and compares them via hypothesis
test: \texttt{anova(lme1,\ lme2)}

To supress refitting with ML (only do this if both models have same
fixed effects): \texttt{anova(lme1,\ lme2,\ refit\ =\ FALSE)}

\subsection{Dropping terms from a
model}\label{dropping-terms-from-a-model}

The \texttt{drop1} function will drop all possible single fixed-effect
terms. We begin with a model and ask, ``what would happen if we dropped
a term''?

Let's say your model is
\texttt{y\ \textasciitilde{}\ x\ +\ r\ +\ z\ +\ (1\textbar{}g)}, fitted
as object \texttt{lme1}.

\texttt{drop1(lme1,\ test="Chisq")} performs a series of Likelihood
Ratio hypothesis tests to see how model is affected by individually
dropping \texttt{x}, dropping \texttt{r}, and dropping \texttt{z}.

Without specifying \texttt{test="Chisq"}, we get a print out of how AIC
changes when terms are dropped.

\subsection{Comparing models with different random
effects}\label{comparing-models-with-different-random-effects}

The typical test is whether or not a random effect is necessary. This
means testing if the variance of a random effect is 0. But variances are
positive and 0 is at the boundary of the range of possible values. The
result is that the standard hypothesis test (ie, a likelihood ratio
test) is \emph{conservative}. The p-value is too high. If you have a
small p-value (say \textless{} 0.001), that's not a problem. If you have
a p-value close to significance, (say about 0.10) you may want to
consider calculating a corrected p-value using a \emph{mixture of
chi-square distributions}.

A likelihood ratio test (LRT) statistic has a chi-square distribution
with degrees of freedom equal to the difference in parameters between
two models. Let's say our LRT is on 2 degrees of freedom. The null
distribution of this test statistic is NOT a chi-square with 2 degrees
of freedom since our null value (variance = 0) is on the boundary of the
parameter space. It's been suggested that a 50:50 mixture,
\(0.5\chi_{df}^{2} + 0.5\chi_{df-1}^{2}\), can serve as a reference null
distribution for computing the p-value. But this is still only an
approximation.

\textbf{Example}: Say we fit two models:

\texttt{lmm1\ \textless{}-\ lmer(wt\ \textasciitilde{}\ weeks\ +\ treat\ +\ (1\ \textbar{}\ subject),\ data=ratdrink)}
\texttt{lmm2\ \textless{}-\ lmer(wt\ \textasciitilde{}\ treat\ +\ weeks\ +\ (weeks\ \textbar{}\ subject),\ data=ratdrink)}

Is the \texttt{weeks} random effect necessary?

\texttt{aout\ \textless{}-\ na.omit(anova(lmm1,\ lmm2,\ refit\ =\ FALSE))\ \#\ drop\ NAs}

\begin{Shaded}
\begin{Highlighting}[]
\CommentTok{\# function for 50:50 mixture}
\NormalTok{pvalMix }\OtherTok{\textless{}{-}} \ControlFlowTok{function}\NormalTok{(stat,df)\{}
  \FloatTok{0.5}\SpecialCharTok{*}\FunctionTok{pchisq}\NormalTok{(stat, df, }\AttributeTok{lower.tail =} \ConstantTok{FALSE}\NormalTok{) }\SpecialCharTok{+}
    \FloatTok{0.5}\SpecialCharTok{*}\FunctionTok{pchisq}\NormalTok{(stat, df}\DecValTok{{-}1}\NormalTok{, }\AttributeTok{lower.tail =} \ConstantTok{FALSE}\NormalTok{)}
\NormalTok{\}}
\end{Highlighting}
\end{Shaded}

\texttt{pvalMix(aout\$Chisq,\ aout\$\textasciigrave{}Chi\ Df\textasciigrave{})}

\end{document}
